\chapter{Daftar Notasi}

Dalam buku ini terdapat beberapa notasi matematika yang digunakan. Para mahasiswa diharapkan dapat memahami notasi-notasi tersebut agar dapat lebih memahami buku ini. Beberapa notasi tersebut adalah sebagai berikut.

\begin{enumerate}
	\item $ \mathbb{N} $ menyatakan himpunan bilangan asli, yaitu $ \lrbr{1, 2, 3, \dots} $\footnote{Terkadang ada beberapa penulis yang memasukkan 0 sebagai anggota bilangan asli.}.
	\item $ \mathbb{Z} $ menyatakan himpunan bilangan bulat, yaitu $ \lrbr{\dots, -2, -1, 0, 1, 2, \dots} $.
	\item $ \mathbb{N}_{0} $ menyatakan himpunan bilangan asli serta angka 0.
	\item $ \mathbb{Z}^{+} $ menyatakan himpunan bilangan bulat positif.
	\item $ \mathbb{Z}^{-} $ menyatakan himpunan bilangan bulat negatif.
	\item $ \mathbb{Q} $ menyatakan himpunan bilangan rasional, yaitu bilangan yang dapat dinyatakan dalam bentuk $ \frac{a}{b} $ dengan $ a $ dan $ b $ bilangan bulat serta $ b \ne 0 $.
	\item $ \mathbb{R} $ menyatakan himpunan bilangan real, yaitu bilangan yang dapat dituliskan dalam bentuk desimal.
\end{enumerate}