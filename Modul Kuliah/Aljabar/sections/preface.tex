\chapter{Kata Pengantar}

Puji syukur kehadirat Tuhan Yang Maha Esa, karena atas rahmat dan hidayah-Nya penulis dapat menulis dan menyelesaikan modul ini dengan senang hati. Beragam halangan dan rintangan telah dilalui dalam proses pembuatan modul pembelajaran ini, dengan harapan mahasiswa program studi Pendidikan Matematika, terutama mahasiswa baru dapat terbantu dalam memahami mata kuliah Aljabar ini.
\par Di Prodi Pendidikan Matematika ini, terutama untuk mata kuliah Aljabar, referensi yang dapat digunakan untuk mahasiswa masih kurang. Mahasiswa sebenarnya dapat mencari di internet mengenai materi aljabar. Tetapi yang penulis lihat di internet, informasi yang diberikan ada yang kurang relevan, bahkan penulisannya pun tidak diformat dengan baik sehingga hal ini juga akan membuat minat mahasiswa turun untuk mencari materi Aljabar. Oleh karena itu, penulis membuat modul ini agar dapat memudahkan mahasiswa untuk mengulas kembali materi-materi yang telah dipelajari sebelumnya. Selain itu, mahasiswa juga dapat menggunakan modul ini sebagai persiapan kuis, ujian tengah semester, maupun ujian akhir semester.
\par Buku ini dibuat dengan menggunakan \LaTeX\ untuk memudahkan dalam penulisan formula matematika. Penggunaan \LaTeX\ juga sangat simpel untuk pembuatan modul seperti ini, karena \textit{template} yang sangat bervariasi di internet. Pembuatan modul ini juga menjadi sarana bagi penulis untuk dapat berlatih dalam menggunakan \LaTeX\ lebih lanjut. Oleh karena itu, apabila terdapat kesalahan dalam buku ini, penulis akan sangat senang mendengar kritik dan saran dari pembaca. Kritik dan saran dari pembaca itulah yang membuat buku ini akan menjadi lebih baik lagi kedepannya.
\par Penjelasan dalam buku ini dibuat sedetail mungkin sehingga mungkin bukan juga sebagai modul, tetapi juga dapat digunakan sebagai buku penuntun pembelajaran. Penulisan yang detail juga mungkin akan membuat pembaca kesulitan dalam memahami tulisan dalam buku ini. Untuk menghadapi hal ini, penulis juga membuat penjelasan yang detail itu dalam bentuk kalimat yang sederhana dan mudah dicerna. Oleh karena itu, jika ada suatu istilah yang kurang familiar, penulis juga terkadang membuat catatan kaki untuk memperjelas makna dari istilah tersebut.
\par Selain penjelasan yang mendetail, dalam modul ini juga terdapat soal-soal latihan yang dapat dikerjakan oleh pembaca agar dapat lebih memahami materi yang diajarkan. Soal-soal latihan yang terdapat pada buku ini dibagi menjadi dua klasifikasi, yaitu soal-soal rutin dan soal-soal tidak rutin (soal tantangan). Tujuannya adalah, agar pembaca selain lebih mengetahui konsep-konsep aljabar, pembaca juga dapat mengasah kemampuannya agar lebih mahir dalam bermatematika. Selain itu, dengan mengerjakan soal-soal yang beragam, pembaca juga dapat mempersiapkan diri untuk menghadapi ilmu matematika yang lebih mendalam.

\vspace{1cm}

\hspace{7.5cm} Anggara Duta Medika