\chapter{Pendahuluan}
\label{sec:intro}

Dalam modul aljabar ini, terdapat tiga materi yang akan dibahas secara mendalam. Ketiga materi tersebut adalah persamaan dan fungsi kuadrat, pertidaksamaan satu variabel, serta eksponen dan logaritma. Sebelum mempelajari lebih mendalam mengenai ketiga materi tersebut, mahasiswa diharapkan mengetahui terlebih dahulu operasi-operasi aljabar serta simbol-simbol matematika, terutama himpunan. Pengetahuan yang cukup mengenai operasi-operasi aljabar akan mempermudah mahasiswa dalam memahami buku ini, bahkan hingga semester-semester berikutnya. Sedangkan pengetahuan yang cukup mengenai himpunan akan membantu mahasiswa dalam memahami materi pertidaksamaan.

\section{Operasi-operasi Aljabar}

Aljabar selalu membahas mengenai simbol-simbol matematika dan aturan-aturannya dalam memanipulasi simbol-simbol ini. Dalam aljabar, ada yang disebut sebagai variabel dan konstanta. Variabel adalah sesuatu yang dapat berubah-ubah. Variabel biasanya disimbolkan dengan huruf-huruf seperti $ x $, $ y $, $ \alpha $, $ \xi $, dan lain sebagainya. Sedangkan konstanta adalah suatu nilai yang tetap seperti 1, 2, $ \pi $, 0, dan bilangan-bilangan lainnya.

\begin{explbox}
	Misalnya kita diberikan $ x = 5 $, apakah $ x $ suatu konstanta atau variabel?
\end{explbox}

\par Terdapat empat operasi dasar aljabar, yaitu penjumlahan, pengurangan, perkalian, dan pembagian. Meskipun dalam buku teks matematika lebih lanjut penjumlahan dan pengurangan dianggap sama, serta perkalian dan pembagian juga dianggap sama, tetapi dalam buku ini kita anggap mereka semuanya berbeda agar lebih simpel dan mudah dipahami.

\par Terdapat istilah matematika juga yang perlu diketahui, yaitu ekspresi aljabar. Ekspresi aljabar adalah kombinasi simbol-simbol yang biasanya terdiri dari gabungan antara variabel, konstanta, dan operasi-operasi aljabar seperti penjumlahan, pengurangan, perkalian, pembagian, dan lain sebagainya. Contoh dari ekspresi aljabar ini adalah $ 2 \cdot x $, yang terdiri dari konstanta, variabel, dan operasi perkalian yang dinotasikan dengan tanda titik ditengah ($ \cdot $). Dari sini, kita bisa yakinkan diri kita bahwa konstanta dan variabel tidak dapat disatukan tanpa suatu operasi aljabar. Oleh karena itu, pembahasan mengenai operasi-operasi aljabar ini sangat diperlukan.

\par Contoh lain dari ekspresi aljabar adalah $ 2x + 3y $. Operasi-operasi ini bisa dipecah menjadi dua bagian, yaitu $ 2x $ dan $ + 3y $. Bagian-bagian inilah yang biasa disebut sebagai suku-suku dari suatu ekspresi aljabar\footnote{Meskipun sebenarnya $ 2x + 3y $ juga dapat disebut satu suku jika kita memisalkan $ u = 2x + 3y $.}. Tetapi, biasanya tanda tambah ($ + $) tidak diikutsertakan dalam pemecahan tersebut sehingga biasanya dituliskan sebagai $ 3y $ saja. Perhatikan bahwa suku-suku adalah pemecahan suatu ekspresi aljabar terhadap operasi penjumlahan dan pengurangan. Oleh karena itu, $ 2x $ disini tidak dipecah lagi menjadi dua ekspresi berbeda (yaitu 2 dan $ x $). Angka 2 pada ekspresi $ 2x $ dan angka 3 pada ekspresi $ 3x $ disebut sebagai koefisien. Bisa dibilang bahwa koefisien ini adalah angka yang berada di "depan" variabel, meskipun sebenarnya adalah angka yang dikalikan dengan suatu variabel karena angka 7 pada ekspresi $ x \cdot 7 $ juga disebut sebagai koefisien karena sesungguhnya $ x \cdot 7 = 7x $.

\par Dalam materi ini, mahasiswa diasumsikan telah mengerti mengenai operasi-operasi aljabar dan tata cara pengoperasiannya. Tetapi perlu digaris bawahi kesalahan-kesalahan yang sering dilakukan oleh mahasiswa dalam mengoperasikan suatu ekspresi aljabar, yaitu sebagai berikut:
\begin{enumerate}
	\item $ \sqrt{x + y} \stackrel{?}{=} \sqrt{x} + \sqrt{y} $;
	\item $ \dfrac{1}{x + y} \stackrel{?}{=} \dfrac{1}{x} + \dfrac{1}{y} $;
	\item $ \left(x + y\right)^{2} \stackrel{?}{=} x^{2} + y^{2} $
	\item $ \dfrac{x}{a} - \dfrac{2x + y}{a} \stackrel{?}{=} \dfrac{-x + y}{a} $; dan
	\item $ \dfrac{2x}{x} \stackrel{?}{=} 2 $.
\end{enumerate}

\begin{explbox}
	Dapatkah Anda memberikan suatu penjelasan mengenai kesalahan-kesalahan ini? Lalu bagaimana penulisan yang benar?
\end{explbox}

\section{Himpunan}

Sampai saat ini, belum ada definisi yang tepat untuk himpunan sehingga secara matematis, himpunan tidak didefinisikan. Meskipun demikian, terdapat beberapa definisi yang tidak formal mengenai himpunan. Bisa dikatakan bahwa himpunan adalah koleksi dari objek-objek yang memiliki karakteristik tertentu yang jelas seperti himpunan mahasiswa Pendidikan Matematika yang mengambil mata kuliah Aljabar. Oleh karena itu, kumpulan dari objek-objek yang belum jelas (masih merupakan pendapat pribadi atau persepsi seseorang) bukan termasuk dalam suatu himpunan, sebagai contoh kumpulan orang-orang yang tinggi bukan merupakan himpunan karena setiap orang memiliki persepsi tersendiri mengenai ketinggian seseorang.

\par Himpunan biasanya dilambangkan oleh huruf besar (misalnya $ A $, $ B $, $ \Gamma $), huruf bergaris ganda (misalnya $ \mathbb{R} $ yang melambangkan himpunan bilangan real dan $ \mathbb{Z} $ yang melambangkan himpunan bilangan bulat), atau huruf berkaligrafi (misalnya $ \mathcal{P} $ dan $ \mathcal{R} $). Sementara itu, anggota dari suatu himpunan biasanya dilambangkan oleh huruf kecil seperti $ a $, $ x $, dan $ \delta $. Keanggotaaan suatu objek dilambangkan dengan tanda "$ \in $" yang pertama kali diperkenalkan oleh Peano (1889) dan lambang untuk ketidakanggotaan adalah "$ \notin $". Misalnya, jika kita diberikan suatu himpunan $ A = \lrbr{c, d} $, maka $ c \in A $ dan $ a \notin A $.

\par Kita tidak membicarakan secara mendetail mengenai himpunan karena akan dibahas mendalam pada mata kuliah Pengantar Dasar Matematika. Pada bagian ini kita hanya membahas mengenai dasar-dasar mengenai himpunan dan notasi-notasinya serta cara mengoperasikan notasi-notasi tersebut.

\par Himpunan dapat direpresentasikan dengan dua cara, yaitu dengan cara enumerasi/pendaftaran dan dengan menggunakan notasi pembentuk himpunan. Pada cara enumerasi, semua elemen (anggota himpunan) dituliskan dan diletakkan diantara kurung kurawal. Sebagai contoh, himpunan bilangan prima yang lebih kecil dari 10 bisa dituliskan sebagai $ \lrbr{2, 3, 5, 7} $. Sementara itu, dengan menggunakan notasi pembentuk himpunan, diantara kurung kurawal diletakkan variabel, kemudian dibatasi dengan tanda "$ | $" dan dituliskan sifat yang mengatur variabel tersebut. Sebagai contoh, himpunan bilangan prima yang lebih kecil dari 10 dapat dituliskan sebagai $ \set{x}{x \mbox{ prima}, x < 10} $.

\par Kadangkala, terdapat suatu himpunan yang tidak memiliki anggota, seperti himpunan semua mahasiswa Hukum yang mengambil mata kuliah Aljabar. Berkaitan dengan hal ini, kita tuliskan himpunan yang tidak memiliki anggota tersebut dengan simbol $ \emptyset $ atau $ \lrbr{} $.

\begin{explbox}
	Diberikan himpunan $ A = \lrbr{\emptyset} $. Apakah $ A $ himpunan kosong?
\end{explbox}

\par Terdapat tiga operasi yang bisa dilakukan pada himpunan, yaitu gabungan, irisan, dan selisih\footnote{Sebenarnya masih banyak lagi operasi-operasi pada himpunan seperti \textit{symmetric difference}. Untuk persiapan menghadapi mata kuliah Aljabar ini, operasi tersebut tidak terlalu sering digunakan sehingga dilewatkan pembahasannya.}. Ketiga operasi tersebut didefinisikan sebagai berikut.

\paragraph{Definisi 1.2.1 \label{def121}} Misalkan $ A $ dan $ B $ himpunan. Maka,
\begin{enumerate}
	\item Gabungan dari $ A $ dan $ B $ yang dilambangkan dengan $ A \cup B $ didefinisikan sebagai
	\[ A \cup B \coloneqq \set{x}{x \in A \, \vee \, x \in B}. \]
	Oleh karena itu, $ x \in A \cup B $ jika dan hanya jika $ x \in A $ atau $ x \in B $. Ilustrasinya adalah sebagai berikut.
	\begin{figure}[H]
		\centering
		\begin{venndiagram2sets}
			\fillA \fillB
		\end{venndiagram2sets}
		\caption{Ilustrasi gabungan dua himpunan.}
	\end{figure}
	\item Irisan dari $ A $ dan $ B $ yang dilambangkan dengan $ A \cap B $ didefinisikan sebagai
	\[ A \cap B \coloneqq \set{x}{x \in A \, \wedge \, x \in B}. \]
	Oleh karena itu, $ x \in A \cap B $ jika dan hanya jika $ x \in A $ dan $ x \in B $. Ilustrasinya adalah sebagai berikut.
	\begin{figure}[H]
		\centering
		\begin{venndiagram2sets}
			\fillACapB
		\end{venndiagram2sets}
		\caption{Ilustrasi irisan dua himpunan.}
	\end{figure}
	\item Selisih dari $ A $ dan $ B $ yang dilambangkan dengan $ A \backslash B $ (atau kadang $ A - B $) didefinisikan sebagai
	\[ A \backslash B \coloneqq \set{x}{x \in A \, \wedge \, x \notin B}. \]
	Oleh karena itu, $ x \in A \backslash B $ jika dan hanya jika $ x \in A $ dan $ x \notin B $. Ilustrasinya adalah sebagai berikut.
	\begin{figure}[H]
		\centering
		\begin{venndiagram2sets}
			\fillANotB
		\end{venndiagram2sets}
		\caption{Ilustrasi selisih dua himpunan.}
	\end{figure}
\end{enumerate}

\par Sebagai contoh, untuk mengiris himpunan $ A = \set{x}{1 < x \leq 5} $ dan $ B \ \set{x}{x > 3} $ bisa dilakukan dengan mencari suatu himpunan $ C $ sedemikian sehingga setiap anggota di $ C $ merupakan anggota dari himpunan $ A $ dan juga merupakan anggota himpunan $ B $. Himpunan $ C $ yang memenuhi ini pastilah $ C = \set{x}{3 < x \leq 5} $. Jadi $ C = A \cap B $. Sedangkan, jika kita ingin menggabungkan $ A $ dan $ B $, maka kita perlu mencari himpunan $ D $ sedemikian sehingga setiap anggota himpunan $ D $ juga berada di $ A $ atau $ B $. Tentunya himpunan $ D $ yang memenuhi adalah $ D = \set{x}{x > 1} $.

\par Terkadang akan sangat sulit untuk mencari himpunan yang memenuhi $ A \cup B $ atau $ A \cap B $ pada contoh di atas. Oleh karena itu, kita bisa menggunakan garis bilangan real untuk mempermudah pencarian gabungan dan irisan dari dua atau lebih himpunan. Jika $ A $ dan $ B $ digambarkan pada garis bilangan real, maka tampaknya adalah sebagai berikut.
\begin{figure}[H]
	\centering
	\begin{tikzpicture}
		\begin{axis}[
			axis x line=middle,
			axis y line=none,
			axis line style=<->,
			xmin=0,
			ymin=0,
			height=2.5cm,
			width=8cm
			]
		{
			\addplot[domain=1:5, thick, blue] {1};
			\addplot[color=blue, holdot] coordinates{(1, 1)};
			\addplot[color=blue, soldot] coordinates{(5, 1)};
			
			\addplot[domain=3:7, thick, red, ->] {2};
			\addplot[color=red, holdot] coordinates{(3, 2)};
		}
		\end{axis}
	\end{tikzpicture}
	\caption{Ilustrasi $ A $ (biru) dan $ B $ (merah) jika digambarkan pada garis bilangan real.}
\end{figure}
Pada gambar di atas, $ A \cup B $ adalah segala garis bilangan yang dilewati oleh garis biru (himpunan $ A $) atau garis merah (himpunan $ B $) sehingga $ A \cup B = \set{x}{x > 1} $. Sementara itu, $ A \cap B $ adalah segala garis bilangan yang dilewati oleh garis biru (himpunan $ A $) dan garis merah (himpunan $ B $) sekaligus sehingga $ A \cap B = \set{x}{3 < x \leq 5} $.

\par Perhatikan sekali lagi pada gambar garis bilangan di atas. Jika titik ujungnya masih termasuk ke dalam suatu himpunan, maka titik ujung garis tersebut harus diberi tanda bulat penuh. Tetapi jika titik ujungnya tidak termasuk ke dalam suatu himpunan, maka titik ujung garis tersebut harus diberi tanda bulat takpenuh. Seperti pada himpunan $ A $ (garis biru), karena $ 5 \in A $ dan 5 adalah titik ujung (kanan) himpunan $ A $, maka garis yang mengilustrasikan himpunan $ A $ pada garis bilangan real diberi tanda bulatan penuh. Begitu juga sebaliknya, karena $ 1 \notin A $ dan 1 adalah titik ujung (kiri) himpunan $ A $, maka garis yang mengilustrasikan himpunan $ A $ pada garis bilangan real diberi tanda bulatan takpenuh.

\begin{explbox}
	Suatu himpunan juga dapat dinyatakan dalam notasi interval. Coba cari tahu mengenai notasi interval tersebut!
\end{explbox}

\par Dalam bekerja dengan menggunakan himpunan, kita harus mengetahui sedang membicarakan apa. Pada penjelasan-penjelasan mengenai himpunan sebelumnya, seperti pada himpunan $ A $ dan $ B $ pada contoh di atas, kita tidak diberikan informasi mengenai variabel $ x $. Kita tidak mengetahui apakah $ x $ termasuk bilangan real, bilangan rasional, atau bilangan bulat. Oleh karena itu, selanjutnya kita harus menspesifikasikan variabel yang bekerja pada suatu himpunan. Jika $ x $ merupakan bilangan bulat, maka penulisan himpunan yang benar adalah $ A = \set{x \in \mathbb{Z}}{1 < x \leq 5} $. Sementara itu, jika $ x $ merupakan bilangan real, maka penulisan himpunan yang benar adalah $ A = \set{x \in \mathbb{R}}{1 < x \leq 5} $. Begitu pula jika $ x $ rasional, $ x $ bilangan asli, atau yang lainnya.

\subsection{Latihan Soal 1.2}
\begin{enumerate}[leftmargin=*]
	\item Buatlah tabel mengenai seluruh kemungkinan interval dan himpunan yang berkaitan beserta ilustrasi pada garis bilangan realnya.
	\item Jika $ A = \set{x \in \mathbb{R}}{x \leq 1} $ dan $ B = \set{x \in \mathbb{Z}}{-2 < x < 3} $, maka tentukanlah $ A \cup B $, $ A \cap B $, dan $ A \backslash B $.
	\item Diberikan himpunan $ \mathcal{P} $ dan $ \mathcal{Q} $. Buatlah suatu ilustrasi mengenai $ \mathcal{P} \backslash \mathcal{Q} $ yang digambarkan pada garis bilangan real untuk tiga himpunan $ \mathcal{P} $ dan $ \mathcal{Q} $ yang berbeda.
	\item Jika $ A $ dan $ B $ adalah sebarang himpunan takkosong, mungkinkah $ A \cup B $ merupakan himpunan kosong? Berikan suatu contoh mengenai hal ini.
	\item Tentukan interval paling sederhana yang ekuivalen dengan
	\[ \left(\left(-15, 20\right) \cup \lrbr{20}\right) \backslash \left(\left[4, 5\right] \cap \lkrb{2, +\infty}\right). \]
	Ilustrasikan pekerjaan Anda dengan membuat interval-interval tersebut pada garis bilangan real.
\end{enumerate}

\section{Identitas Aljabar}

Akan sangat membantu jika Anda mengetahui identitas-identitas aljabar dalam mata kuliah ini. Identitas-identitas aljabar ini dapat membantu meringankan beban Anda dalam menghitung suatu ekspresi matematika yang rumit. Berikut merupakan identitas-identitas yang sering digunakan untuk membantu menyelesaikan permasalahan matematika.
\begin{enumerate}
	\item $ x^{2} + y^{2} = \left(x + y\right)^{2} - 2xy = \left(x - y\right)^{2} + 2xy = \dfrac{1}{2}\left(\left(x + y\right)^{2} + \left(x - y\right)^{2}\right) $
	\item $ x^{2} - y^{2} = \left(x + y\right)\left(x - y\right) $
	\item $ x^{3} + y^{3} = \left(x + y\right)\left(x^{2} - xy + y^{2}\right) $
	\item $ x^{3} - y^{3} = \left(x - y\right)\left(x^{2} + xy + y^{2}\right) $
	\item Untuk setiap bilangan bulat positif ganjil $ n $,
	\[ x^{n} + y^{n} = \left(x + y\right)\left(x^{n - 1} - x^{n - 2}y + \cdots - xy^{n - 2} + y^{n - 1}\right). \]
	\item Untuk setiap bilangan bulat positif genap $ n $,
	\[ x^{2} + y^{n} = \left(x - y\right)\left(x^{n - 1} + x^{n - 2}y + \cdots + x^{n - 2}y + y^{n - 1}\right). \]
	\item Untuk setiap bilangan bulat positif $ n $,
	\[ x^{n} - y^{n} = \left(x - y\right)\left(x^{n - 1} + x^{n - 2}y + \cdots + xy^{n - 2} + y^{n - 1}\right). \]
	\item $ x^{3} + y^{3} + z^{3} - 3xyz = \left(x + y + z\right)\left(x^{2} + y^{2} + z^{2} - xy - yz - xz\right) = \dfrac{1}{2}\left(x + y + z\right)\left(\left(x - y\right)^{2} + \left(y - z\right)^{2} + \left(z - x\right)^{2}\right) $
	\item Untuk setiap bilangan bulat positif $ n \geq 2 $,
	\[ \left(x_{1} + x_{2} + \cdots + x_{n}\right)^{2} = x_{1}^{2} + x_{2}^{2} + \cdots + x_{n}^{2} + 2\left(x_{1}x_{2} + x_{2}x_{3} + \cdots + x_{n - 1}x_{n} + x_{n}x_{1}\right). \]
	\item $ \left(ax + by\right)^{2} \pm \left(ay \mp bx\right)^{2} = \left(a^{2} \pm b^{2}\right)\left(x^{2} \pm y^{2}\right) $
	\item $ x^{2}y + y^{2}z + z^{2}x + x^{2}z + y^{2}x + z^{2}y + 2xyz = \left(x + y\right)\left(y + z\right)\left(z + x\right) $
	\item $ a^{4} + 4b^{4} = \left(a^{2} + 2b^{2} - 2ab\right)\left(a^{2} + 2b^{2} + 2ab\right) $
\end{enumerate}

\subsection{Latihan Soal 1.3}
\begin{enumerate}[leftmargin=*]
	\item Uji kebenaran identitas-identitas pada subbab ini.
	\item Cari tahu mengenai Simon's Favorite Factoring Trick (SFFT). Apa kaitannya dengan identitas-identitas di atas?
	\item Faktorkanlah ekspresi aljabar $ \left(x - y\right)^{3} + \left(y - z\right)^{3} + \left(z - x\right)^{3} $.
\end{enumerate}